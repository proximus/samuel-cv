%===============================================================================
%
%         FILE: cv.tex
%
%        USAGE: ---
%
%  DESCRIPTION: ---
%
%      OPTIONS: ---
% REQUIREMENTS: ---
%         BUGS: ---
%        NOTES: ---
%  ORIG AUTHOR: Samuel Gabrielsson <samuel.gabrielsson@gmail.com>
% ORGANIZATION: ---
%      VERSION: 1.0
%      CREATED: 2021-05-04 13:46:31
%     REVISION: ---
%      CHANGES: ---
%
%===============================================================================
\documentclass[10pt,a4paper,sans]{moderncv}

% Configure
\moderncvstyle{classic}             % Style options: casual, classic, banking, oldstyle and fancy
\moderncvcolor{blue}                % Color options
\usepackage[scale=0.75]{geometry}   % Adjust the page margins

% Personal data
\name{Samuel}{Gabrielsson}
%\title{Resumé Title}
\address{Saitama AB}{Lindhagensgatan 59}{112 43 Stockholm}
\phone[mobile]{+46~(2)72~152~42~28}
\email{samuel.gabrielsson@gmail.com}
%\homepage{www.ludd.ltu.se/~proximus}

\photo[64pt][0.4pt]{profile-picture}

% Social icons
\social[linkedin]{samuelgabrielsson}
\social[github]{proximus}

% BibTex adjustments
%   - Show numerical labels in the bibliography
\renewcommand*{\bibliographyitemlabel}{[\arabic{enumiv}]}

\begin{document}
%===============================================================================
% Resume/CV Content
%===============================================================================
\makecvtitle

%===============================================================================
% Work experience
%===============================================================================
\section{Experience}
\cventry{2018 -- Ongoing}{Chief Executive Officer (CEO)}{Saitama AB}{Stockholm}{Sweden}{
Owner and founder of the consultant company Saitama AB.
\begin{itemize}
\item \textbf{\textit{Consultant at Ericsson AB - Systems and Technology (R\&D)}}\newline{}
Platform research and development of 5G and 6G Testbeds.
\end{itemize}
}
\bigskip

\cventry{2016--2018}{Senior Software Architect and Design}{Realtime Embedded AB}{Stockholm}{Sweden}{
\begin{itemize}
\item \textbf{\textit{Consultant at NorthStar}}\newline{}
Development of NorthStar Advanced Connected Energy (ACE) gateway smart battery
system.
    \begin{itemize}
    \item Implement external alarm handling for the battery system. Ensure that
          the software can trigger and control relays through the GPIO pins in
          the gateway when correct signaling through IPC messages are received.
    \item Implement functional tests and libraries/packages using the 9pm test
          framework.
    \end{itemize}
\item \textbf{\textit{Consultant at Vanderbilt (former Bewator)}}\newline{}
Development of Linux Kernel drivers for access control system called
Omnis. The hardware used was called E100 which is developed on site
and is similar to AT91SAM9x5-EK.
    \begin{itemize}
    \item Write GPIO test application to control relays.
    \item Write Atmel AT91 ADC test application to read voltage raw values.
    \item Write Linux kernel and device tree patch to enable USB-B gadget and
          Ethernet over USB.
    \item Write Linux kernel driver and application for 74VHC595 8-bit shift
          register with output latches in order to control via sysfs LEDs, input
          ports and RS485 card readers.
    \item Write Linux kernel driver and test application for Magstripe
          (Clock/Data) and Wiegand (Data/Data) card readers.
    \item Write a firmware upgrade/flash script to automatically update the
          system using the SD/MMC card during bootup.
    \item Modify and patch the bootloader from Atmel called at91bootstrap in
          order to fix bugs and enable bootup of SD/MMC or NAND flash.
    \end{itemize}
\end{itemize}
}
\bigskip

\cventry{2011--2016}{Team Leader/Software Developer}{Ericsson AB}{Stockholm}{Sweden}{
Employed at Ericsson in Kista as a Software Developer and Team Leader. I
developed drivers for the radio hardware units used in the Radio Base Station
6000 series. In order to first boot and bring up the radio hardware, the
different drivers for the many components on the board needs to be working and
configured correctly. This is done by either creating the drivers from scratch
using the components data sheet and the boards circuit diagram or simply by
adapting and reusing existing drivers. In most cases we emulate the hardware
first and then write the drivers, before the physical boards arrive from the
factory, to save time and money.
    \begin{itemize}
    \item Development of radio software drivers for the Radio SW platforms and
          Radio Hardware emulation. Board bringup and integration of hardware
          products like Antenna Integrated Radio unit (AIR), Micro Radios,
          Next Generation Radios and the 5G Program. Other side tasks include
          splitting parts of the Radio SW code into modules, introducing Git,
          Gerrit, Jenkins, BitBake and Autotools.
    \item Short term assignment (3 months) at the hardware department in Ottawa
          as a software integrator for Platform 5.2 AIR 32 products.
    \item Act as a technical Scrum/Kanban Master to lead and coach a
          Cross-Functional Driver team of 10 people. Conduct daily scrum
          meetings, sprint retrospective meetings, keep statistics of flow,i
          eliminate waste and help Product Owner.
    \end{itemize}
}
\bigskip

\cventry{2008--2011}{Software Developer}{SAAB Combitech}{Stockholm}{Sweden}{
My first consultant job started at SAAB Combitech for Ericsson AB Linux Center
in Älvsjö and Telefonplan. There, I developed a test framework in Perl which is
used to run automated test cases. The test cases was written in Tcl/Expect but
the framework can also execute test cases in Perl or Bash and also support
NetConf. The test framework became a separate product and was used to test
different Linux embedded operating systems by many Ericsson products at the
department. I continued in the same department but as a designer where I began
to develop Linux kernel patches and pushed upstream to Linus Torvalds.
    \begin{itemize}
    \item Develop drivers and Linux kernel patches for the Ericsson Linux
          distribution called ENUX, which is used to run a Virtual Machine to
          emulate APZ classic run time environment for PLEX based programs such
          as AXE.
    \item Designer and tester for other Ericsson platform products such as CPBS,
          GSDI and IS running on GEP board hardware.
    \item Framework developer for automated test cases.
    \item Develop automated test cases for Ericsson products Linux Open-source
          Telecom Cluster (LOTC) and SLES Virtual Platform (SVP) based on Xen.
    \item Open Multimedia Platform (OMP) customer forum support.
    \item Handle and solve trouble reports for all the products.
    \end{itemize}
}
\bigskip

\cventry{2006--2007}{Research Assistant}{University of Toronto}{Toronto}{Canada}{
I worked as a research assistant for professor J. Christopher Beck at the
Department of Mechanical and Industrial Engineering. My stay there was funded
by a scholarship from the University of Toronto (UofT) and Toronto Intelligent
Decision Engineering Laboratory (TIDEL). During my research, I became extremely
interested in operations research, optimization and solving NP-hard
combinatorial problems with the help of distributed algorithms. Together with
another PhD student, we published a paper\cite{DuaGab2007} and the whole
experience abroad gave me new friends for life, deepened my analytical thinking
and broaden my knowledge in research and software development.
    \begin{itemize}
    \item Research in artificial intelligence and planning. Published paper:
          "Modelling Security Protocol Synthesis with AI Planning", UofT TIDEL,
          ON, Canada.
    \item Research in optimization, parallel and distributed algorithms and
          communications protocol. Published paper: "Solving Combinatorial
          Problems with Parallel Cooperative Solvers", Ninth International
          Workshop on Distributed Constraint Reasoning, Brown University
          Providence, Rhode Island, USA.
    \item MPI Cluster Computing, installation and configuration.
    \end{itemize}
}
\bigskip

%===============================================================================
% Misc activities/jobs
%===============================================================================
\subsection{Miscellaneous}
\cventry{2006--2006}{Committee Member}{Lule\aa \ University Academic Computer Society}{City}{Lule\aa}
{Served 600 registered student members.}
\cventry{2004--2005}{Treasurer}{Lule\aa \ University Academic Computer Society}{City}{Lule\aa}
{Responsible for a budget of over 180 000 SEK.}
\bigskip

\cventry{2000--2001, Summers 2002--2005}{Mail Carrier}{City-Mail}{Stockholm}{Sweden}{
    \begin{itemize}
    \item Sorting and distribution.
    \item Updating the computer-based estate register.
    \end{itemize}
}
\bigskip

\cventry{1999--2000}{Substitute Teacher}{S\"odert\"alje Municipality}{S\"odert\"alje}{Sweden}
{Teaching junior high school students math, chemistry and physics.}
\bigskip

\cventry{1999--1999}{Military Service, KA1 Waxholm}{Swedish Government}{Stockholm}{Sweden}{
    \begin{itemize}
    \item The royal guards at the royal palace in Stockholm and security guard.
    \item Leadership experience and basic medical education.
    \item Handle technical equipment for advanced management.
    \end{itemize}
}

%===============================================================================
% Education
%===============================================================================
\section{Education}
\cventry{2001--2006}{M.Sc. Computer Science \& Engineering}{Luleå University of Technology}{Luleå}{Sweden}
{Specialized in Applied Mathematics}
\subsection{Master thesis}
\cvitem{title}{\emph{A Parallel Tabu Search Algorithm for the Quadratic Assignment Problem}}
\cvitem{supervisors}{Professor J. Christopher Beck - University of Toronto \& Professor Inge Söderkvist - Luleå University of Technology}
\cvitem{description}{Find optimal solutions for a quadratic assignment problem using the tabu search algorithm on a cluster.}
\bigskip

\cventry{1995--1998}{Natural Science Programme}{V\"asterg\aa rd High School}{S\"odert\"alje}{Sweden}{}

%===============================================================================
%  Skills
%===============================================================================
\section{Skills}
\cvskilllegend*[1em]{}  % Adjust post spacing
\cvskillhead[-0.1em]    % Insert standard legend in English and adjust padding

% Programming
\cvskillentry*{Programming Languages:}{3}{Ansible}{2}{I use Ansible, YAML and Redfish for zero-touch automation on SuperMicro and Dell PowerEdge machines that has BMC/IPMI/iDRAC to configure BIOS, autoinstall and setup real-time OS.}
\cvskillentry{}{2}{Assembler}{1}{Mips32 course at university.}
\cvskillentry{}{5}{Bash}{23}{Used in most of my work projects and at home to quickly write scripts with minimum dependencies. Also used together with Sed, Awk and Regular Expressions.}
\cvskillentry{}{4}{C/C++}{15}{I have used C/C++ in most of the big Linux embedded projects, Linux applications, kernel modules and patches.}
\cvskillentry{}{2}{CSS}{2}{Used to develop my personal homepage a long time ago.}
\cvskillentry{}{3}{Haskell}{2}{I learned recursive programming at university.}
\cvskillentry{}{3}{HTML}{2}{Used to develop my personal homepage a long time ago.}
\cvskillentry{}{3}{Java}{2}{From different courses at university.}
\cvskillentry{}{4}{\LaTeX}{5}{I've used \LaTeX{} in everything from thesis report, research papers, my company annual reports to this CV.}
\cvskillentry{}{3}{Listp/Scheme}{2}{Did the SICP course at University and to configure my Emacs editor.}
\cvskillentry{}{3}{Maple}{2}{Used in mathematics courses at the university.}
\cvskillentry{}{3}{Matlab}{5}{Used in mathematics courses at the university and while working at Ericsson research.}
\cvskillentry{}{3}{Pascal}{3}{Self-taught Turbo Pascal as a child.}
\cvskillentry{}{4}{Perl}{7}{Wrote a test framework at Ericsson Linux DC that executed Perl and Expect test cases.}
\cvskillentry{}{3}{PHP}{2}{Used to develop my personal homepage a long time ago.}
\cvskillentry{}{4}{Python}{15}{I wrote a test framework and corresponding test cases. Python has also been used for misc scripting in bigger projects throughout the years.}
\cvskillentry{}{3}{Ruby}{2}{Used to write test cases at Radio SW. Very nice and friendly scripting language with a lot of handy predefined functions.}
\cvskillentry{}{5}{Tcl/Expect}{13}{I sometimes use Expect for automation and test cases in different projects.}

% Software Tools
\cvskillentry*{Tools:}{3}{Buildroot}{3}{Used to build embedded Linux OS.}
\cvskillentry{}{3}{Docker}{3}{Used when isolating applications at Ericsson Testbed.}
\cvskillentry{}{3}{ClearCase}{3}{Version control system used at Ericsson but replaced by Git.}
\cvskillentry{}{5}{Gerrit}{10}{Code reviewing tool used in most of my projects together with Git and Jenkins.}
\cvskillentry{}{5}{Git}{13}{Simply the best version control system out there.}
\cvskillentry{}{4}{Jenkins}{10}{Combined with Gerrit and Git, you get the best tool for continuous integration.}
\cvskillentry{}{3}{SVN}{1}{Version control system that I used when working at Vanderbilt.}
\cvskillentry{}{5}{Vim}{20}{My main editor used for coding and editing.}
\cvskillentry{}{3}{Yocto/BitBake/Poky}{3}{Used to build embedded Linux OS.}

% Operating Systems
\cvskillentry*{OS:}{5}{Linux}{23}{Starting with RedHat, Slackware, Linux From Scratch and settling for Ubuntu.}
\cvskillentry{}{3}{OSE}{5}{The main OS in the old Ericsson Radio product.}
\cvskillentry{}{3}{Solaris}{5}{Primary OS at the university.}
\cvskillentry{}{3}{Windows/MS-DOS}{25}{Dual booting Windows only when playing PC games at rare occasions.}

\cvskillentry*[1em]{Methods}{4}{Kanban/Scrum}{5}{Team leader for Radio Software Board team.}

\section{Languages}
\cvitemwithcomment{Swedish}{Native speaker}{Born and raised in Sweden}
\cvitemwithcomment{English}{Fluent}{Thanks to early training since primary school}
\cvitemwithcomment{Aramaic}{Fluent}{Ancient christian language spoken by my parents}
\cvitemwithcomment{German}{Fair}{Learned in school}

%===============================================================================
% Reference
%===============================================================================
%\begin{category}{References}
%\citemnobullet Will be supplied on request.
%\citem{Vice-chancellor Inge S\"oderkvist} (Thesis supervisor) \\
%Lule\aa\ University of Technology \\
%Department of Mathematics \\
%S-97187 Lule\aa, Sweden \\
%\ \\
%Phone: +46 (0)920 492130 \\
%Fax: +46 (0)920 491073 \\
%Email: inge@sm.luth.se \\
%Web site: http://www.ltu.se/staff/i/ingsor

%\citem{Professor J. Christopher Beck} \\
%University of Toronto \\
%Department of Mechanical \& Industrial Engineering and \\
%Department of Computer Science \\
%5 King's College Rd. \\
%Toronto, ON M5S 3G8, Canada \\
%\ \\
%Phone: +1 416 946 8854 \\
%Fax: +1 416 878 7753 \\
%Email: jcb@mie.utoronto.ca \\
%Web site: http://www.mie.utoronto.ca/staff/profiles/beck
%\end{category}

%===============================================================================
% Publications
%===============================================================================
\nocite{*}
\bibliographystyle{plain}
\bibliography{publications} % publications is the name of a BibTeX file

%===============================================================================
% Research Projects
%===============================================================================
%\begin{category}{Research \\ Projects}
%\citembullet ``Modelling Security Protocol Synthesis with AI Planning'', University of Toronto, ON, Canada (December 2006).
%\citembullet ``Density Functional Theory and a Simulation of Water Absorption on a Germanium Surface'', Lule\aa\ University of Technology, Sweden (March 2005).
%\end{category}

\end{document}
